\documentclass[tikz]{standalone}


\begin{document}

\node (MR) {Mundo real}
  child {node [descricao] {É o local tridimensional em que vivemos e habitamos todos os dias}};

\node (MT) [below of=MR] {Mundo da tela}
  child {node [text width = 3cm] {Refere-se às telas bidimensionais onde assitimos imagens}
    child (AT) {node [descricao] {O mundo da imagem da alta tecnologia que criamos com câmeras e computadores}}
		child (BT) {node [descricao] {O mundo da imagem da baixa tecnologia que criamos com lápis e pincéis}
    }
	};

\node (PI) [below of=MT] {Plano da imagem}
  child {node [descricao] {O plano da imagem é uma janela bidimensional imaginária. \\ Seu tamanho depende da distância da cena à câmera}
    child {node [descricao] {Toda tela é um plano de imagem. As proporções variam de acordo com o contexto.}}
  }
  child {node {Planos}
    child {node {Primeiro Plano (FG)}}
    child {node {Plano Intermediário (MG)}}
    child {Plano de Fundo (BG)}
  };

\end{document}
