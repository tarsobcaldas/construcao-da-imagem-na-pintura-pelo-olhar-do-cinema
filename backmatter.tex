\chapter{Considerações Finais}%
\label{consideracoes}

Chega a hora de conciliar imagens e palavras a um ponto final. Rever o
percurso desta investigação desde quando decidi abraçar a causa da
pintura e me debruçar sobre inquietações de ordem pessoal, relacionadas
a duas artes bidimensionais: cinema e pintura. A escrita acadêmica,
para artistas, encontra temas de interesse não só ligados a questões de
ordem teórica sobre as humanidades, mas que com frequência se
entrelaçam ao seu trabalho prático.

Ao buscar responder à pergunta sobre como a tela branca pode ser
abordada como forma de expressão pessoal, cujo principal interesse eram
teorias de narrativa visual cinematográfica, trilhei um caminho que
passou pela revisão histórica das primeiras teorias da imagem em
movimento, já muito ligadas à história da pintura da época. Encontramos
a contextualização sobre o cinema em Xavier, 1983, que define a
estrutura e organização do filme como imagem e som \emph{organizados de
	um certo modo}. A \emph{impressão de realidade} é a tônica de um cinema
ficcional que, segundo ele, pouco mudou em seus princípios, no período
entre 1916 e 1980. Analisamos as relações do cinema com vanguardas
europeias em movimentos como o Impressionismo, Futurismo,
Expressionismo e Dadaísmo.

Cinema e pintura foram os dois campos pesquisados, cuja motivação se
iniciou nos anos 1980 com a experiência da câmera de filmar Super 8,
durante minha graduação em cinema, passando pelos estudos de pintura na
EAV e EBA, até chegar neste mestrado. Considerei a necessidade de
atualizar o referencial teórico, escolhendo autores ainda atuantes nas
academias, como Jacques Aumont, André Gardies e Delfim Sardo para
compor a perspectiva da investigação, que se firmou com o argumento de
Eric Rohmer de que o cinema é, acima de tudo, a arte do espaço. Com a
\emph{companhia} e o auxílio de historiadores da arte, teóricos,
curadores e pintores, realizei uma revisão bibliográfica
teórico/prática, a fim de responder três das minhas principais
indagações:

\begin{itemize}
	\item Qual a possibilidade de se utilizar o olhar característico do cinema,
	      no método processual da construção de uma pintura?

	\item Como se movimentam os olhos dos espectadores diante de uma pintura?

	\item Como é feita a seleção, pelos pintores, das referências que apoiarão
	      seus projetos?
\end{itemize}

Para tal finalidade, retomei as teorias de Jacques Aumont sobre
representação, espaço e quadro, relacionadas à produção de atmosferas
realistas no cinema. O texto de Aumont nos leva a confirmar a primeira
característica do olhar do cinema, que também diz respeito à pintura: o
quadro. 

Levando em conta estes aspectos, podemos dizer que questões
sobre a representação do espaço no quadro, seja na pintura à época de
Leonardo, seja no estilo centralizado \emph{hollywoodiano}, que reproduzia
técnicas antigas de centro de simetria, permanecem atuais, ainda que
tenhamos novas formas de visualização como IMAX, 3D, entre outras, que
se distanciam da noção convencional de ecrã no cinema e quadro na
pintura. As redes sociais como o \emph{Instagram} apresentam evidências da
relevância de questões relacionadas ao quadro, ainda hoje.

Ainda sobre o quadro, com o desdobramento das teorias de Aumont ao
abordar termos como centralização\slash descentralização e
enquadramento/desenquadramento, forneceram o entendimento da natureza
do problema sobre os passos iniciais da pintura diante da tela branca.
Tratava-se de decisões relacionadas não ao tema e à narrativa, mas
sobre o recorte da imagem, a relação com a borda. O pensamento sobre a
imagem que seria pintada se baseava no posicionamento da câmera para
realizar o enquadramento ou desenquadramento, que eu praticava com a
minha câmera de Super-8. Importava muito menos a história a ser contada
que os limites da tela, que me negava a múltiplas opções oferecidas
naturalmente pela intermediação da câmera. Desenhar um objeto na tela,
ou localizar este objeto na tela de pintura segundo regras de percepção
propagadas por Arnheim não solucionava o problema de reproduzir a forma
natural com a qual eu pensava o sangramento de uma imagem, ou seja,
pelo corte, evidenciado especialmente na obra Madra inglesa, única
pintura da série que intitulei de Escorço. Será necessário encenar esta
mala sequencialmente, agora com o devido esclarecimento dado pelo
confronto das teorias de Bordwell e Arnheim sobre as centralizações
hollywoodianas, bem como diante dos textos de Bonitzer sobre
descentralização versus desenquadramento. Considero, assim, importante
no seguimento desta pesquisa, revisar experimentações e processos para
uma pintura figurativa, descritos no \cref{cap4-entre-projeto-processo}, trazendo novamente
para o ateliê a interlocução entre a fita crepe e o projetor. A
familiarização com elementos em cena, proporcionada pelos estudos com a
figura da mala pode resultar em novos caminhos processuais, mas este é
um tema a considerar para uma nova investigação.

Um segundo ponto conclusivo, ainda ligado à primeira pergunta de
investigação, parte das teorias de André Gardies, sobre a alternância
da posição do sujeito inscrito na ordem cotidiana que cede seu lugar ao
sujeito espetacular, devido às restrições físicas da sala do
espetáculo. Considerando este aspecto, ouso traçar aqui o
relacionamento com a ideia imersiva, derivada da projeção de imagens
para realização de esboços. O espaço aqui é entendido como um lugar, no
ateliê de pintura, que acolhe o pintor no momento de reconstrução de
uma imagem dada.

Um lugar que inscreve o sujeito pintor como espectador, e vai além se
colocando do outro lado da cena, também como autor de um espaço
diegético. Um terceiro ponto, que talvez não se relacione diretamente
com a construção da pintura pelo olhar do cinema, mas com o espaço
expositivo, objetivo final da fatura, vem da ideia de ambiente tratada
por \textcite{sardo2017exercicio}, sobre a expansão das artes de projeto. Como já
mencionado, refleti: Como estaria sendo transformada a atmosfera
fílmica na caixa preta e nas imagens da pintura no cubo branco depois
do advento do celular? O espaço da tela do cinema na sala escura e da
pintura na galeria tem hoje uma outra configuração, ao se tornar móvel
e transitar por outros lugares. Percebo hoje uma expansão das artes, em
suas diversas modalidades, para um outro espaço, intermediado por telas
de celulares e tantos outros dispositivos, evocando o espaço diegético,
que, conforme define Gardies, é duplamente construído: pelo próprio
filme e pelo espectador. Existe um novo espaço expositivo, de galerias,
museus, feiras de arte e sites de artistas, nascido no ciberespaço,
cuja imprevisibilidade de recepção por parte do espectador se torna um
novo desafio para o autor, tanto da pintura tradicional quanto do
cinema.

Sardo, lembra a experiência de Aby Warburg e seu Atlas Mnemosyne nas
constelações de sentido, além da a ideia de intervalo trazida da
fotografia em séries. O cinema, dedica um espaço importante à criação
por meio de sequências, na montagem da estrutura narrativa visual dos
filmes, conforme vimos em Block. Com isto, encontramos subsídios para
desenvolver o conceito de serialidade, que importa muito no processo de
construção de projetos pautados pela similaridade, em vários aspectos,
mas também pela restrição e pelo sentido. Trazemos aqui o exemplo na
pintura do trabalho de Yvens Klein, com suas pinturas restritas à cor
azul.

Para entender como se movimentam os olhos dos espectadores diante de
uma pintura, visitamos referências sobre a óptica e movimento na
pintura em uma perspectiva de ordem prática. Trouxemos para esta
conversa o livro de David Hockney, transformado em documentário, o de
Majevski sobre o Brugel, o Velho e a influência dos estudos de
movimento de Da Vinci na obra de Rubens, e a apropriação dos autores
contemporâneos, Muntean e Rosenblum. Se ainda hoje estudos apontam para
evidências de influências entre dois grandes pintores que tinham o
movimento como traço característico de sua obra, como Rubens e Da
Vinci, encontramos em autores contemporâneos, como Muntean \& Rosenblum,
de modo semelhante, uma forma de representar e direcionar o movimento
do olhar, que se nutre de referências de autores de tempos passados. Em
virtude do que acabamos de mencionar, podemos dizer que a arte
contemporânea acumula muitas camadas da história do movimento na
pintura, que se mesclam, em seu desdobramento, com a história do
cinema.

A estrutura proposta por Block para a narrativa visual proporciona um
entendimento simples de um item complexo da estruturação da imagem: a
perspectiva. A sua aplicação pode ser experimentada hoje facilmente no
ecrã dos celulares e transposta para a pintura de diversas formas. Ao
atualizar as estratégias da narrativa visual de Block, o estudo e uso
de termos como espaço profundo no cinema, foi possível identificar em
meu trabalho de pintura a recorrência de diagonais cruzando a tela.
Diagonais estas que, anteriormente, eu denominava genericamente de
perspectiva. Em consequência da revisão destas teorias aplicadas à
sétima arte, se tornou claro o fato de que eu já vinha usando este
olhar característico do cinema em meu trabalho.

No \cref{cap3-narrativa-visual} investiguei como é feita a seleção, pelos pintores, das
referências que apoiarão seus projetos. Ao analisar a trajetória da
pintora brasileira Lucia Laguna, observei em sua fala o uso permanente
de referências, tanto de artistas pelos quais se interessa, quanto do
conceito de acúmulo de imagens que remete à colagem. O pensamento sobre
a colagem criou nexos entre o movimento em direção à planaridade na
pintura de Bruegel e suas múltiplas narrativas em rede. Já na pintura
de Cristina Canale, cenas do cotidiano que partem de imagens de
revistas de moda e redes sociais, remetem a uma motivação pessoal que
traz um outro campo de estudo para dentro da sua poética. Ao querer
compreender as ligações entre o cinema e a minha pintura, encontrei na
obra de Cristina Canale uma forma semelhante de atuar diante de
referências ao trazer um outro campo de estudo para dentro da sua
poética, que no caso são elementos da moda. Fundamentada na composição
da forma com elementos da moda, que aliam figuração e abstração
costuradas pela cor, sua obra aponta para a viabilidade de um projeto
artístico também fundamentado na janela que tenho chamado de olhar do
cinema. Tendo em vista estes aspectos, concluo que utilizar processos
cinematográficos para realizar uma pintura é uma forma de absorver
referências. São recortes do mundo real intermediado por uma câmera ou
já tratados anteriormente por outras mídias, que escolho para reeditar
na pintura. Percebo que, com isto, ocorre uma interligação de olhares
distintos.

O distanciamento necessário à pesquisa científica dura pouco para a
artista. Escrever sobre o que nos afeta termina em um mergulho em nosso
próprio trabalho. O detalhamento das questões relativas ao movimento
dos olhos, dentro do plano cinematográfico, já tratados no capítulo ii,
me fez pensar em um projeto de pintura inspirado na estrutura visual em
questão. Ou seja, a ideia de partir do cinema para a pintura, por meio
dos elementos categorizados por Block. O objetivo de aplicar alguns
conceitos da narrativa visual do filme e criar de séries de pintura que
se relacionassem com a pesquisa, proporcionou um estudo metodológico de
todo o trabalho plástico realizado desde o início do mestrado. A partir
disto, foi possível definir processos que se adequassem à elaboração de
diferentes projetos.

Assuntos explorados nesta dissertação, como por exemplo o do espaço
profundo no cinema, vincularam-se ao projeto expositivo pessoal
intitulado Olhar do cinema, que abordei no \cref{cap4-entre-projeto-processo}. Coletar e
organizar imagens, selecionar referências e aprofundar conceitos que
pretendia abordar me fez compreender uma forma prática de lidar com um
repertório de ideias, por vezes amplo, mas em outros momentos,
restrito. Encontrei na serialidade das sequências um dos temas do
projeto, que também serve de ferramenta para projetos que se constroem
pela similaridade, pela restrição e pelo sentido. Ao criar uma espécie
de laboratório de experimentação baseado em termos do cinema (lembrando
Aby Warburg e seus protocolos de montagem do Atlas Mnemosyne),
redescobrimos possibilidades de expansão e restrição, onde os
agrupamentos de temas se expandiram e contraíram tal qual uma rede, uma
teia. O uso de influências de outros artistas permeou tanto as técnicas
dos esboços realizados quanto a reafirmação de temas do cinema, por
meio da série Humano em frames. Refletimos sobre a marcação de esboços
por meio de projetores para iniciar um trabalho pictórico, um recurso
usado por pintores contemporâneos, como alternativa aos modelos vivos e
técnicas de ampliação de desenho. Podemos afirmar que esta parte do meu
processo criativo acontece de maneira indireta, onde os modelos do
mundo real já sofreram um tratamento por outros meios de representação
da imagem, no caso a fotografia e o cinema. Além disto, foi inevitável
a associação do prazer de pintar ao da imersão em um espaço de exibição
de filmes. Pela observação destes aspectos, acredito ser relevante
propor e disseminar métodos alternativos para o ensino acadêmico de
pintura, alinhados à poética de artistas que trabalham no âmbito da
arte contemporânea. Este período de imersão na pesquisa bibliográfica,
artistas de referências pesquisados e do próprio trabalho em ateliê,
permitiu comprovar que um pensamento pictórico pode partir não só do
estudo da linha, da forma e da mancha no desenho, mas também ter sua
origem em outras formas de se pensar a imagem: como a fotografia, o
cinema, a música, a instalação, a arquitetura. Meios que existem hoje
para auxiliar a expressão e instauração da fala de artistas, em
especial a do pintor.

Hoje continuo com a intenção de construir uma imagem diretamente no
espaço de uma tela de pintura, me valendo de um pensamento acostumado a
olhar através de uma câmera. No entanto, a bagagem que se acumulou na
mala é muito maior, depois destes três anos de mestrado. O projeto
ainda se encontra em andamento. Mas trago a certeza de que neste
processo, conforme fala
\parencite{sabino2015pintura}, não posso abdicar do envolvimento com a
matéria da tinta.


\printbibliography[title=Referências]
