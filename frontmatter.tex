% LTeX: enabled=false
\cover

\titlepage

\thispagestyle{empty}


\chapter*{Agradecimentos}

\thispagestyle{empty}

Ao Carlos Caldas, o Charlie, meu companheiro e amor desta vida, pelo
incentivo e apoio incondicional a este e a todos os projetos artísticos
que tenho realizado.

Aos meus filhos queridos, Julia e Tarso, que me ajudam a atualizar o
olhar e ter esperança em um mundo melhor.

À minha orientadora Joana Rego, que acompanhou esta pesquisa desde seu
processo embrionário,~com as primeiras pinturas em Portugal, me
apresentando e inspirando com referências importantes, como~a dos
escritos inacabados de Walter Benjamin, deixados em uma mala.

À minha coorientadora India Mara Martins, especialmente pelo acolhimento
na disciplina Teorias da imagem e a construção do olhar, no PPGCine-Uff,
um momento importante para a retomada dos estudos e da escrita
vinculados ao cinema e audiovisual.

À Rosa e Ana Maria que além de cuidar das condições físicas do Ateliê de
Pintura da Boa Hora, nos apoiavam afetivamente garantindo a paz
necessária a este espaço compartilhado de criação.

À Isabel Barroso, pela sua disponibilidade em orientar sobre as
ferramentas disponíveis para a formatação deste trabalho, junto ao SDI
da Biblioteca da \ac{fbaup}.

Aos professores e orientadores das instituições, cursos e consultorias,
que iluminaram este percurso, representados aqui por Alberto Fernandes, Ana Kothe, Caio
Pacela, João Magalhães, Jozias Benedito, Luiz Ernesto, Luana Manhães,
Monique Queiroz, Shannon Botelho e Rafael Bteshe.

Aos amigos do cinema e da pintura, do Brasil e de Portugal, e não
somente estes, que se fazem representar aqui pelos que me apoiaram de
perto neste percurso, como Damian Ross, Glaucia Xavier, Lena Mendes, Marcela Pedersen, Marcia
Campbell, Maria de Fátima Ginicolo, Stella Margarita, Maia Bueloni,
Raquel Dora Pinho, Ricardo Larangeira e Taty Arruda.

\clearpage

\thispagestyle{abstract}

\begin{abstract}
	O debate sobre como a imagem é construída e se apresenta no mundo
	atual, está cada vez mais presente em muitos campos de investigação. O
	crescente interesse de pessoas em se iniciar nas artes visuais, cuja
	cultura visual não se relaciona diretamente ao mundo da pintura,
	justifica a escolha deste tema. Com a pergunta inicial sobre como a
	tela branca será abordada para expressar nossas próprias ideias,
	objetivamos investigar como utilizar~o olhar característico do cinema,
	no método processual da construção de uma pintura, ou de outra forma,
	no exercício da atividade do pintor. Partindo do contexto histórico do
	nascimento do cinema, foi realizada uma pesquisa bibliográfica
	envolvendo os autores Jacques Aumont, André Gardies, Delfim Sardo, e
	outros que, em suas abordagens, vinculam o olhar característico do
	cinema e da pintura a conceitos de movimento, espaço e \emph{óptica}.
	Investigamos o~uso~de referências para projetos artísticos pessoais,
	com origem em outros campos, por meio da trajetória de duas pintoras
	brasileiras: Lucia Laguna e Cristina Canale. A metodologia de nossa
	pesquisa foi a conjugação de uma revisão bibliográfica com a
	experimentação, de ordem processual e prática, realizada no ateliê do
	Mestrado em Artes Plásticas da Faculdade de Belas Artes da \ac{uporto}. A
	partir do estudo realizado neste trabalho e, nas reflexões sobre o uso
	de projeção de imagens na realização de esboços, concluímos ser
	relevante propor e disseminar métodos alternativos ao ensino acadêmico
	de pintura, alinhados à poética de artistas que trabalham no âmbito da
	arte contemporânea.

  \medskip
	\noindent\emph{\bfseries Palavras-chave:} Construção da imagem; olhar; cinema; pintura; movimento; espaço.
\end{abstract}

\clearpage

\thispagestyle{abstract}

\renewcommand{\absnamepos}{flushright}
\begin{otherlanguage}{english}
	\begin{abstract}

		The debate on how the image is constructed and presented in today's
		world is increasingly present in many fields of investigation. The
		growing interest in embarking on
		visual arts, whose visual culture is not directly related to the world
		of painting, justifies the choice of this theme. Starting with the question 
    of how the white canvas will
		be approached to express our own ideas, we  investigate how to
		use the characteristic visual style of cinema, whether, in the
		procedural method of the construction~of a painting or, 
    in the exercise of the painter's activity. We provide
		historical context from the
		birth of~cinema, through authors such as Jacques Aumont, André Gardies,
		Delfim Sardo, and others who, in their approaches, link the
		characteristic looks of cinema and painting to concepts of movement,
		space and optics. We then investigate the use of references for
		personal artistic projects, originating in other fields, through the
		trajectory of two Brazilian painters: Lucia Laguna and Cristina Canale.
		Our research methodology 
		combined this bibliographic review with experimentation, of a
		procedural and practical nature, carried out in the studio of the
		Master's in Plastic Arts at the Faculty~of Fine Arts
    in \ac{uporto}. From the study carried out in this work and in our
		reflections on the use of~image projection in the creation of sketches,
		we conclude that it is relevant to propose and disseminate alternative
		methods to the academic teaching of painting, in line with the poetics
		of artists who work in the field of contemporary art.

    \medskip
		\noindent\emph{\bfseries Keywords:} Image construction; view; cinema; painting; movement; space.
	\end{abstract}
\end{otherlanguage}

\clearpage

\tableofcontents*

\clearpage

\listoffigures*

\clearpage


\listofquadros*


\chapter*{Lista de Abreaviatura}

\begin{acronym}
	\acro{apa}[APA]{Associação Americana de Psicologia}

  \acro{app}[App]{Aplicativo}

	\acro{eav}[EAV]{Escola de Artes Visuais do Parque Lage}

	\acro{fbaup}[FBAUP]{Faculdade de Belas Artes da Universidade do Porto}

	\acro{ppgcine-uff}[PPGCine-UFF]{Programa de pós-graduação em cinema e
		audiovisual da Universidade Federal Fluminense}

	\acro{sdi}[SDI]{Sistema de Documentação e Informação}

	\acro{ufrj}[UFRJ]{Universidade Federal do Rio de Janeiro}

	\acro{uporto}[UPORTO]{Universidade do Porto}
\end{acronym}

\clearpage

\begin{authornote}
	Esta dissertação foi escrita em conformidade com a norma técnica da
	APA, 7\textordfeminine~edição. Diagramação foi feita em \LaTeX\ por Tarso Boudet Caldas e 
  o design da capa por Adriana Tavares.

	Ao buscar um consenso entre acordos ortográficos sobre a palavra
	óptico/ótico, relativo ao olho, encontrei informações sobre a grafia da
	palavra óptica \emph{s.f.}~ciência da visão e ótica \emph{s.f.}~ciência
	da audição. Dicionários reconhecidos, inseridos no contexto do Acordo
	Ortográfico, divergem quanto à grafia da consoante p, por ser muda na
	pronúncia individual dos vários falantes de língua portuguesa. Em
	Portugal não se usa óptica, enquanto no Brasil são aceitas as duas
	grafias, embora mantenha-se a distinção entre os significados
	referentes a visão e audição \parencite{ILLLP2015optico}.

	Diante das várias possibilidades de escrita para uma única palavra,
	deixo de lado as discussões que insistem em permanecer, referentes à
	língua que compartilhamos neste espaço de escrita e opto por adotar o
	português materno, do Brasil, com o qual me expresso desde que nasci.
	Assim espero comunicar melhor as imagens, que tanto importam nesta
	pesquisa.

  As citações dos textos de \citeauthor{gardies2019espaco} e 
  \citeauthor{block2013visual} foram traduzidas pela a autora, 
  e possuem indicações semelhantes às notas de rodapé, porém com 
  letras do alfabeto. Os textos originais correspondentes podem ser encontrados
  na seção \hyperref[sec:textos-originais]{\enquote{Textos originais}}.
\end{authornote}
